\begin{samplecase}
{\bf Maxwellian averaged cross section at 30 keV: n + ${}^{138}$Ba}\newline
In this sample case, the Maxwellian averaged cross section (MACS) at 30 keV is calculated.
The following input file is used

\VerbatimInput{\samples n-Ba138-MACS/org/talys.inp}

where the line {\bf astroE 0.03} determines the energy in MeV for which the
average is calculated. Note that the line {\bf energy 1.} is irrelevant, 
but must nevertheless be given.
The file {\em astrorate.g} looks as follows:
{\small \begin{verbatim}

# header:
#   title: Ba138(n,g) reaction rate
#   source: TALYS-2.0
#   user: Arjan Koning
#   date: 2023-12-14
#   format: YANDF-0.1
# target:
#   Z: 56
#   A: 138
#   nuclide: Ba138
# reaction:
#   type: (n,g)
#   Q-value [MeV]:  4.723434E+00
#   E-threshold [MeV]:  0.000000E+00
#   <E> [MeV]: 3.000000E-02
#   astrogs: y
# datablock:
#   quantity: reaction rate
#   columns: 4
#   entries: 1
##       T        reaction rate      MACS           G(T)
##   [10^9 K]      [cm3/mol/s]       [mb]            []
   3.481369E-01   8.521989E+05   5.887537E+00   1.000000E+00
\end{verbatim} } \renewcommand{\baselinestretch}{1.07}\small\normalsize
\noindent
in which the temperature, reaction rate and MACS is given.
For this case, the deviation from a normal 30 keV calculation (which can be 
obtained by removing the {\bf astroE} line and using {\bf energy 0.03}) is 
small.
\end{samplecase}
